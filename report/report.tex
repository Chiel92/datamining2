\title{Lab Assignment 2 \\ \small{Datamining}}
\author{Chiel ten Brinke 3677133}
\documentclass[12pt]{article}
\usepackage{amssymb,amsmath,amsthm,enumerate,graphicx,float,lmodern}

\newtheorem{theorem}{Theorem}[section]
\newtheorem{lemma}[theorem]{Lemma}
\newtheorem{proposition}[theorem]{Proposition}
\newtheorem{corollary}[theorem]{Corollary}

\theoremstyle{definition}
\newtheorem{definition}[theorem]{Definition}
\newtheorem{axiom}[theorem]{Axiom}
\newtheorem{example}[theorem]{Example}
\newtheorem{remark}[theorem]{Remark}

\newcommand{\set}[2]{\left\lbrace#1 \, \middle|\, #2 \right\rbrace}

\begin{document}
\maketitle

\section*{Data Analysis}
\label{sec:data_analysis}

\subsection*{a}
There are 10 variables, so there are $\frac{10 \cdot 9}{2} = 45$ undirected edges.
Since each combination of edges represents exactly one graphical model,
there are $2^{45} = 35184372088832$ possible graphical models.

\subsection*{b}
The table of counts has a cell for each combination of variable values.
This comes down to $9 \cdot 2 \cdot 2 \cdot 2 \cdot 3 \cdot 6 \cdot 4 \cdot 3 \cdot 5 \cdot 2 = 155520$ possible combinations.
This is confirmed by taking the length function of the table of counts in R.
The number of parameters of the saturated model equals the number of cells in the table of counts.

\subsection*{c}
Cliques: 


\end{document}
